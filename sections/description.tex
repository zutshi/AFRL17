\section{Project Description}

%What is the problem
%
%Why is the problem imp
%
%Application of prior falsification work to UxAS architecture
%
%Why can I investigate the problem
%    Falsification expertise - citations

\subsection{Introduction}

We propose to explore falsification techniques for the analysis of
hybrid systems. Using our previous work and developed tools, we
propose to address both safety and security analysis of software
controlled hybrid dynamical systems. We expect the research to yield
techniques which can be used to analyze relevant subsystems of UxAS
and other generic cyber-physical systems.

Specifically, we are interested in improving the applicability of
existing automatic falsification methods and investigate their
applicability to finding security vulnerabilities of controllers under
attack. We assume the model of the system under test/attack as a
sampled data control system (SDCS), where the controller program
samples and actuates a   black model of the plant (a hybrid
dynamical system) periodically. Furthermore, we want to benchmark such
grey-box falsification techniques on UxAS sub-systems. To accomplish
these objectives, we propose the below directions.
\begin{itemize}
    \item \emph{Automatic falsification}: Apply grey-box falsification
        techniques to UxAS subsystems.
    \item \emph{Plant Models}: Investigate standardized models for
        hybrid dynamical systems which are conducive to falsification
        analysis. The models will be constructed using the available
        black box plant simulators or data.
    \item \emph{Security Analysis}:
        Transform the problem of searching for an attack signal which
        can take the system to unsafe states, into a falsification
        problem.
\end{itemize}

Automatic falsification has been explored in our previous
work~\cite{Zutshi+Others/2013/Trajectory, zutshi2014multiple,
zutshi2016symbolic}. Construction of piece-wise affine (PWA) plant
models for reachability analysis has been explored
in~\cite{Zutshi2016} with preliminary results. Such models are
amenable to symbolic analysis and can be generated automatically or
semi-automatically
Furthermore, symbolic models can be composed with
control code and model checked using mature model
checkers~\cite{kroening2014cbmc}. Security analysis for linear systems
has been explored in~\cite{pajic2014robustness, pajic2015attack}.

%The accuracy of the PWA models can
%be controlled by several parameters including the discretization step
%$\Delta_t$, quantization function $Q$ and number of samples.

%Falsification techniques for security analysis.

% involves the complex
% interaction between software and physical world, but also missing
% analysis for considering software artifacts along with plant models
% - The models are often not there or not detailed enough verify
% properties at hand

% % In the specific context problem
% UAV subsystems testing procedure

% -
% - Verifying low level controllers against physical dynamics- subsystem
% falsification: but does not compose. Module level verification

% proposal:
%     - pwa modeling
%         - Can be generated quickly
%         - Existing methods to check
%     - Verifying low level controllers against physical dynamics -
%       subsystem falsification: but does not compose.
%     - security?


\subsection{Background and Challenges}

Reachability analysis of hybrid systems is a undecidable for all but
the simplest kind~\cite{henzinger1995s}. Existing analysis can be
partitioned into verification~\cite{tiwari2012hybridsal,
Chen2012taylor, Frehse+Others/2011/SpaceEx, althoff2016combining,
duggirala2015c2e2} and falsification techniques~\cite{annpureddy2011s,
donze2010breach, dreossi2015efficient}. The verification procedures
use over-approximations to verify a given property whereas current
falsification techniques are comparable to testing methodologies and
search for a concrete violations of the property.

We are interested in analyzing hybrid systems which result due to the
pairing of software controllers with continuous dynamical systems.
These are commonly modeled as sampled data control systems.

\begin{wrapfigure}{r}{0.42\textwidth}
%\vspace{-1.5cm}
%\centering
  \begin{center} \tikzstyle{line} = [draw, very thick, color=black!80, -latex']
  \tikzstyle{block} = [rectangle, minimum width=4.4cm, minimum height=1.5cm, text centered, draw=black, fill=blue!25]
  \tikzstyle{cell} = [rectangle, minimum width=0.7cm, minimum height=0.7cm, text centered, draw=black, densely dashed]
  \tikzstyle{var} = [rectangle, text centered]
  \tikzstyle{arw} = [->, thick,>=stealth,shorten <=2pt, shorten >=2pt]
  \tikzstyle{arw2} = [->, thick,>=stealth, shorten >=2pt]
  \tikzstyle{Arw} = [thick,double,->,shorten <=2pt,shorten >=2pt, >=stealth]
  \tikzstyle{line} = [thick]
  \tikzstyle{abvmid} = [above, midway]

  \def\boxh{1}
  %midpoint
  \def\mdpt{\boxh/2-\boxh/5}
      %TODO: better def of mdpt
    %($(controller) !.5! (plant)$)

  % Relative horizontal position of blocks against controller and plant
  % blocks

  % Column1: Cy, Cx
  \def\colax{1.6}
  % Column2: s, u
  \def\colbx{0.8}
  % Column3: s', u
  \def\colcx{0.8}
  % Column3: Cx', Cy
  \def\coldx{1.6}

  \begin{tikzpicture}[scale=1.0, align=center, on grid, auto],
        %\node (code) [rectangle, draw=black] {Controller\\Code};
      \node (controller) [block]{
	Control Software + IDS\\ (program analysis)
        %\node (controller) [block] {Controller \\
        %$\pgmA:(C^y,\varphi)\mapsto(\psi,U)$
	};

        \node (plant) [block, below = 2.5cm of controller] {
            Plant Model\\(abstract)};
          %$\begin{array}{lcl}
          %  \vx' &=& \simulate(\vx,\vu,\sampletime)\\
          %   \vy &=& g(\vx)\\
          %\end{array}$};

        % ($(controller.west)+(-\colcx,-0.5-\mdpt)$)

    %\draw[Arw] (code) -- (controller) node[midway,right] {Symbolic\\Execution};

    \def\Cypos{($(controller.west)-(\colax,\mdpt)$)}
    %\node (Cy) [cell, left of=controller, node distance=3.0cm] {$y \in C_y$};
    %\node (Cy) at \Cypos [cell] {$C^y$};
    %    \draw[arw] (Cy.east) -- ($(controller.west)-(0,\mdpt)$) node[abvmid]{\scriptsize{\sample}};
    %\node (Cx) [cell, left of=plant, node distance=3.0cm] {$x \in C_x$};
    %\def\Cxpos{($(plant.west)+(-\colax,\mdpt)$)}
    \def\Cxpos{($(plant.west)-(\colax,\mdpt)$)}
    %\node (Cx) at \Cxpos [cell] {$C^x$};
        %\draw[arw] (Cx.east) -- ($(plabvmid.west)-(0,\mdpt)$) nodeabvmid]{\scriptsize{\sample}};
    %\draw[arw] (Cx.east) -- ($(plant.west)-(0,\mdpt)$) node[abvmid]{\scriptsize{\sample}};

    %\node (s) [var, above of=Cy, node distance=1.0cm] {$s$};

    %\node (s) at ($(controller.west)+(-\colbx,\mdpt)$) [var] {$\varphi$};
    %    \draw[arw] (s.east) -- ($(controller.west)+(0,\mdpt)$) node[abvmid] {};

    %\node (u) [var, below of=Cx, node distance=1.0cm] {$\vu$};
    %\def\upos{($(plant.west)-(\colbx,\mdpt)$)}
    \def\upos{($(plant.west)+(-\colbx,\mdpt)$)}
    %\node (u) at \upos [var] {$\vu$};
        %\draw[arw] (u) -- ($(plant.west)+(0,\mdpt)$);
        %\draw[arw] (u.east) -- ($(plant.west)+(0,\mdpt)$) node[midway,above] {sample};

      %\def\Cx_pos{($(plant.east)+(\coldx,\mdpt+0.5)$)}

    %\node (Cx_) at \Cx_pos [cell] {$C^{x'}$};
    %\draw[arw] ($(plant.east)+(0,\mdpt)$) -- (Cx_.west) node[abvmid,sloped]{\scriptsize{$\quant$}};
    %\node (Cy_) [cell, below = 1.5cm of Cx_] {$C^y$};
    %\draw[arw] ($(plant.east)-(0,\mdpt)$) -- (Cy_.west) node[abvmid,sloped]{\scriptsize{$\quant$}};

    %\draw[arw] (Cy_) |- (-4.5,-4) |- (Cy.west);
        %\draw[arw] (Cy_) |- ++(-7.4,-0.8) |- (Cy.west);
    %\coordinate (x) at (left = 1 of Cy, below = 1 of Cy_);
        %\draw[arw] (Cy_) |- (x) |- (Cy.west);
    %\node (CxCymid) at ($(Cx)!0.5!(Cy)$) [] {};
    %\node (Cx_Cy_mid) at ($(Cx_)!0.5!(Cy_)$) []{};
        %\coordinate (Cx_Cy_mid) at ($(Cx_)!0.5!(Cy_)+(0.2,0)$) []{};
    %\coordinate (CxCymid) at ($(Cx)!0.5!(Cy)$) [] {};
    %\coordinate (CxCymid2) at ($(CxCymid) - (0.5,0)$) []{};
    %\node (CxCymid2) at ($(CxCymid) - (0.5,0)$) []{};
    %\draw[line, shorten <=2pt] (Cx_.south) -- (Cx_Cy_mid);
    %\draw[line, shorten <=2pt] (Cy_.north) -- (Cx_Cy_mid);
    %\draw[line] (Cx_Cy_mid) -| ++(0.5,-1.3) -| (CxCymid2);
    %\draw[thick] (CxCymid2) -- (CxCymid);
    %\draw[arw2] (CxCymid2)--(Cy.south);
    %\draw[arw2] (CxCymid2)--(Cx.north);


    \node (control_input) at ($(controller.east)+(\colcx,-0.5-\mdpt)$) [var] {$\vu$};
        \draw[arw] ($(controller.east)$) -- ($(controller.east)+(0.5,0)$) -- ($(plant.east)+(0.5,0.3)$) -- ($(plant.east)+(0,0.3)$);
    \node (omegap) at
    ($(plant.east)+(1.0,-0.3)$) [var] {$\vec{\omega_p}$};
        \draw[arw] ($(plant.east)+(0.5,-0.3)$) --
        ($(plant.east)+(0,-0.3)$);

    \node (output) at ($(controller.west)+(-\colcx,-1.5-\mdpt)$) [var] {$\vy$};
    \node (output_hat) at ($(controller.west)+(-\colcx,0.2-\mdpt)$) [var] {$\hat{\vy}$};
        \draw[arw] ($(plant.west)$) -- ($(plant.west)-(0.5,0)$) -- ($(controller.west)-(0.5,0.3)$) -- ($(controller.west)-(0,0.3)$);

    \node (omegac) at
    ($(controller.west)-(1.0,-0.3)$) [var] {$\vec{\omega_c}$};
        \draw[arw] ($(controller.west)-(0.5,-0.3)$) --
        ($(controller.west)-(0,-0.3)$);

        %\node (attacker) [rectangle, text centered, draw=black, fill=white] at ($(plant.west)+(-0.5,controller.y-plant.y)$) {$+$};
        \node (attacker) [rectangle, text centered, draw=black,
        fill=white] at ($($(plant.west)-(0.5,0)$) !.5!  ($(controller.west)-(0.5,0.3)$)$) {$+$};

    \node (attack) at ($(attacker.east)+(0.7,0)$) [var] {$\vec{a}$};
        \draw[arw] ($(attacker.east)+(0.5,0)$) -- ($(attacker.east)$);

    %\node (s_) at ($(controller.east)+(\colcx,\mdpt)$) [var] {$\psi$};
     %   \draw[arw] ($(controller.east)+(0,\mdpt)$) -- (s_);

    %\node (u_) at ($(controller.east)-(-\colcx,\mdpt)$) [var] {$U$};
        %\draw[arw] ($(controller.east)-(0,\mdpt)$) -- (u_);
        %\draw[arw] (u_) |- ++(-1.0,-0.7) -| (u);
        % TODO: node's position should be relative to u but is hard
        % coded!
        %\draw[arw] (u_) -- ++(0,-0.7) -- ++(-5.0,0) node[midway]{$\scriptsize{\mathsc{sample\_smt}}$} -- (u);

\end{tikzpicture}
\end{center}
\vspace{-0.4cm}
\caption{Closed loop symbolic execution.}
\vspace{-2.8cm}
\label{fig:clse}
\end{wrapfigure}

\mypara{Sampled Data Control System (SDCS)} consists of two
components, as illustrated in Figure~\ref{fig:clse}. (a) A discrete
time plant model $P$ , and (b) a controller implementation $C$
described by a program whose semantics are formally defined.  Finally,
the closed-loop parallel composition assumes that the continuous plant
has been discretized with the controller's sampling period $\Delta_t$.

The SDCS model allows for modeling of measurement noise at the
controller and plant disturbances using exogenous input $\omega_c$ and
$\omega_p$. Additionally, the generic attack model is described as an
additive input $\vec{a}$ to the plant's output $\vy$, at the sensor.

Current testing methodologies for software controlled plants
separately test the software controller and the plant. Although
useful, such strategies do not address the combined functional
behavior. Hence we propose building up on our earlier work of
symbolic-numeric falsification approaches~\cite{zutshi2016symbolic}.
%, augmenting the more rigorous, but less
%scalable verification approaches~\cite{majumdar2012clse}.

PWA models have been proposed in the past for modeling hybrid
dynamical systems (surveyed in~\cite{paoletti2007identification}),
with their verification in~\cite{yordanov2007model,
yordanov2010formal, koutsoukos2003safety, batt2007model}. As the
individual models are linear, they are amenable to formal analysis.
However, recent work on the formal analysis of neural networks has
been explored using SMT solvers in~\cite{pulina2012challenging,
pulina2011never, pulina2011checking, katz2017reluplex}. Due to recent
advances in non-linear SMT solvers, we would like to explore such
models in the present framework of falsification.

%The challenges are as follows
%Apply our previous work~\cite{zutshi2014multiple, zutshi2016symbolic}
%to UxAS. Explore falsification using models
    %Discovered violations must be reproducible. This entails a combined analysis of the software and physical model, as separate analysis will result in false positives.
    %Ability to explore software artifacts which can lead to functional `bugs'.

\subsection{Proposed Research}

In our prior research, we have explored the problem of falsification
in hybrid systems from the perspective of white/grey/black box
testing. In~\cite{Zutshi+Others/2013/Trajectory}, we explored the use
of multiple shooting for re-formulated the reachability decision
problem into the feasibility problem and solved using off-the-shelf
optimization engines. We extended this approach to black box
testing~\cite{zutshi2014multiple} by combining multiple shooting with
a counter-example guided abstraction refinement (CEGAR) procedure.
Finally, we combined symbolic analysis of programs with the analysis
of hybrid systems for falsification analysis of sampled data control
systems in~\cite{zutshi2016symbolic}.
%All the above approaches were prototyped as tools and benchmarked on several case studies.
Using these techniques, we propose to address the safety and security
analysis for control system implementations.

\mypara{Safety Analysis of Sampled Data Control Systems}
We have explored the symbolic-numeric analysis approach towards
falsification of safety properties in ~\cite{}.


%- Falsification of automated co-operative unmanned air vehicle missions

We are exploring relational PWA models of plants, specifically built
for the purpose of reachability analysis.  Relational
abstractions~\cite{Tiwari2012} and our previous
research~\cite{zutshi2012timed}. Treating the system as a black box,
we split the problem into (a) modeling the behavior of the system
using a piecewise affine discrete time model and (b) encoding the
search for falsification as a bounded model checking problem. Such a
model is specialized for the given property.  SMT solvers and
optimization techniques can be used to model check the resulting
system.  Furthermore, we would like to explore models based on machine
learning in the same vein using non-linear SMT
solvers~\cite{gao2013dreal}.

%If found, we check
%the existence of the violation in the original system to ensure
%reproducibility

\mypara{Security Analysis of Sampled Data Control Systems}
\todo{falsification for security}

We propose to extend the falsification framework to find attack
signals. We focus on the most general model where the sensor can be
compromised. The model also includes an intrusion detection system,
which further constraints on the attack signal, thereby focusing the
search for non-trivial attacks.  Using the same framework for search
for exogenous disturbance, we propose to find attack signals.




%   human-machine interaction to decentralized cooperative control
%   automatically construct a surveillance path given high level goals
%   route planning, coordinate behavior among multiple vehicles, connect with external software and hardware devices, validate mission requests, log and diagram message traffic, and optimize task ordering
%   common message passing architecture




% \subsection{Proposed Study}

% The Project Description should provide a clear statement of the work to be undertaken and must include: objectives for the period of the proposed work and expected significance; relation to longer-term goals of the PI's project; and relation to the present state of knowledge in the field, to work in progress by the PI under other support and to work in progress elsewhere.

% The Project Description should outline the general plan of work, including the broad design of activities to be undertaken, and, where appropriate, provide a clear description of experimental methods and procedures.  Proposers should address what they want to do, why they want to do it, how they plan to do it, how they will know if they succeed, and what benefits could accrue if the project is successful. The project activities may be based on previously established and/or innovative methods and approaches, but in either case must be well justified. These issues apply to both the technical aspects of the proposal and the way in which the project may make broader contributions.
